\documentclass{book}

\usepackage{alltt}
\usepackage{calc}
\usepackage{ccicons} % Creative Commons icons
\usepackage[colorlinks,linktocpage]{hyperref}
\usepackage{url}
\usepackage{xeCJK}

\setCJKmainfont{BIZ UDMincho}
\setCJKsansfont{BIZ UDGothic}
\setCJKmonofont{BIZ UDGothic}

\newlength{\negspacelength}
\setlength{\negspacelength}{\bigskipamount+\medskipamount}

\newcommand{\negspace}{\vspace{-\negspacelength}}

\def\whalfspace{\vspace*{0.335\baselineskip}}
\def\wspace{\vspace*{0.67\baselineskip}}
\def\nspace{\vspace*{-.7\baselineskip}}
\def\nspacemargin{\vspace*{-1\baselineskip}}

\begin{document}

\frontmatter

\date{\today}

\title{Relational Interpreters in miniKanren \\ \ \\ (WORKING ROUGH DRAFT --- DRAFT 0)}

\author{William E. Byrd}

\maketitle

{
  \fontsize{12}{12}
  \mbox{}
  \vfill

%  \hrule

\noindent
\textcopyright~2024 William E. Byrd


\noindent
\huge
\ccLogo
\ccAttribution
\large


\noindent
This work is licensed under a Creative Commons Attribution 4.0 International License.
(CC BY 4.0) \\
\url{http://creativecommons.org/licenses/by/4.0/}
}

\newpage

{
\ \\

\ \\

\noindent
\huge
To Dan Friedman
}

\newpage

\tableofcontents

\chapter{Preface}

The intent of this book is to share the techniques, knowledge, pitfalls, open problems, promising-looking future work/techniques, and literature of writing interpreters as relations in miniKanren.
%
Someone who reads this book actively should be ready to understand, implement, modify, and improve interpreters wtitten as miniKanren relations, read the related literature, and perform original research on the topic.

\section{What this book is about}

This book is about writing interpreters for programming languages, especially for subsets of Scheme.
While there are many books on writing interpreters, this book is unusual in that it explores
how to write interpreters as \emph{relations} in the miniKanren relational programming language.
By writing interpreters as relations, and by using the implicit constraint solving and search in the \texttt{faster-miniKanren} implementation, we can use the flexibility of relational programming to allow us to experiment with programs in the language being interpreted.  For example, a relational interpreter can interpret a program with missing subexpressions\footnote{Such programs are often called \emph{program sketches} [TODO cite].}, or \emph{holes}, attempting to fill in the missing subexpressions with values that result in valid programs in the language being interpreted.  Or we can give both a program containing holes and the value we expect the program to produce when interpreted, and let \texttt{faster-miniKanren} try to fill in the holes in a way to produce the expected output.  We can even write an interpreter that explicitly handles errors, and ask \texttt{faster-miniKanren} to find inputs to the program that trigger these errors.\footnote{This is known in the literature as ``angelic execution''.}

\section{What you need to know to read this book}

Although this book contains a brief introduction to Scheme, and an introduction to miniKanren, the book is not intended as a tutorial on the fundamentals of programming, nor as an introduction to functional programming.  Similarly, the book is not intended to be a primer on the fundamentals of programming language theory, design, or implementation.  While I do try to explain important Scheme and programming languages concepts as they arise (such as lexical scope, closures, and environment-passing interpreters), I assume the reader has enough experience and knowledge to follow along with minimal examples and explanations of these fundamental concepts.  If you've encountered these ideas before, and just need a little refresher, I hope the level of explanations and examples will be helpful and sufficient.  If you are familiar with functional programming and interpreters, but don't know Scheme, the examples and explanation should also be helpful and sufficient.  If you are familiar with some version of miniKanren or microKanren, the chapters on miniKanren should be helpful, since we'll be using aspects of the \texttt{faster-miniKanren} implementation of miniKanren that extend (and may  differ from) the languages described in the first and second editions of \emph{The Reasoned Schemer}, the microKanren papers, my dissertation, and other miniKanren literature.

Since I know different readers will be coming to this book with very different backgrounds, I've added ``pretests'' to the Scheme and miniKanren introduction chapters, to help you determine if you already know the concepts well enough to skip ahead.  Even if you are a Scheme expert, you should probably read the section on pattern matching to make sure you understand the syntax and semantics of the pattern-matcher we'll be using.  If you haven't used \texttt{faster-miniKanren} before, or a miniKanren that supports the \verb|=/=|, \verb|symbolo|, \verb|numbero|, and \verb|absento| constraints, I strongly suggest you read the entire introduction to miniKanren.

\section{What is not in this book}

One important topic this book does not cover is how to implement a miniKanren---for example, how \texttt{faster-miniKanren} is implemented.  While this is an interesting topic, and is especially important for some advanced optimizations and for implementing new constraints, this book focuses on writing interpreters as relations.  There are other resouces on implementing simple miniKanrens, such as the papers on microKanren [TODO cite these], which is the basis for the miniKanren implementation in the second edition of \emph{The Reasoned Schemer} [TODO cite].

\section{Running the code in this book}

The code in this book was tested with Chez Scheme and Racket.  It should be possible to run most code in other Scheme implementations, with few or no changes, with the exception of code that makes extensive use of Chez-specific or Racket-specific features, which I will point out in those chapters, as appropriate.

\subsection{Getting \texttt{pmatch} from GitHub}

\subsection{Getting \texttt{faster-miniKanren} from GitHub}

\url{https://github.com/michaelballantyne/faster-miniKanren}

\begin{alltt}
git clone git@github.com:michaelballantyne/faster-minikanren.git
\end{alltt}

Alternatively, you can click on \verb|<> Code| button and select \verb|Download ZIP| to download and uncompress the \texttt{.zip} file containing the entire \texttt{faster-miniKanren} directory.


\subsection{Using this book with Chez Scheme}

\subsubsection{Installing Chez Scheme}

\subsubsection{Starting a Chez Scheme REPL}

\subsubsection{Loading a file in Chez Scheme}

\subsubsection{Loading \texttt{faster-miniKanren} in Chez Scheme}


\subsection{Using this book with Racket}

\subsubsection{Installing Racket}

\url{https://racket-lang.org/}

\url{https://download.racket-lang.org/}

\subsubsection{Important differences between Chez Scheme and Racket}

representation of quoted values

evaluation order

language levels

macros

\subsubsection{The DrRacket IDE and the Racket REPL}

\subsubsection{Starting and configuring DrRacket}

changing default language

changing default memory limit

\subsubsection{Starting a Racket REPL}

\subsubsection{Requiring a module in Racket}

\subsubsection{Requiring the \texttt{faster-miniKanren} module in Racket}

\section{Acknowledgements}

Dan Friedman and Michael Ballantyne both encouraged me to continue working on this book, and independently encouraged me to break down one giant book into more than one book, each book being more manageable.
%
Dan encouraged me to write a short and direct primer on Scheme with only the needed parts of the language.
%
Michael also encouraged me to continue working on the book in the open.

Darius Bacon wrote me a very helpful email about how using two separate lists to represent a lexical environment, rather than a single association list, can result in better performance and divergence behavior.  I had played around with this representation in the past, but had abandoned it before I understood its advantages.  Thank you, Darius.

My mother has continually encouraged me to work on this book, and most importantly, to finish it!

[TODO add other acknowledgements]

[TODO add acks for typesetting tech, such as the fonts; also can add colophon if I'm so inclined]

\mainmatter

\chapter{Enough Scheme to get by} %%%%%%%%%%% intro to Scheme %%%%%%%%%%%

We need to know some Scheme, since Scheme is the host language for the \texttt{faster-miniKanren} version of miniKanren we will be using. \texttt{faster-miniKanren} inherits Schemely features such as \texttt{cons} pairs, \texttt{quote}, and \texttt{letrec}.

We also need to know some Scheme because we will be writing interpreters for subsets of Scheme.  In particular, we need to feel comfortable with the evaluation rules for Scheme, including the notions of expressions and values.

And we need to know some Scheme if we want to be able to read much of the miniKanren literature.\footnote{A reading knowledge of OCaml would also be helpful for reading the miniKanren literature that uses OCanren, a miniKanren-like language embedded in OCaml.}

\section{A few comments on Scheme}

small core

compositional

few exceptions to rules

very powerful---lots of ways to do meta-programming, including the
ability to extend the syntax of the language

great for writing interpreters, compilers, and DSLs

\section{The Scheme reports, versions of Scheme, and implementations of Scheme}

\section{Which version and implementations of Scheme we are using, and why}

\section{What we need to know about Scheme, and when}

\section{Useful Scheme resources}

[todo add full references and URLs; can point to the relevant sections as I describe aspects of Scheme]

The Scheme Programming Language, 4th Edition

The Chez Scheme User's Guide [TODO check spelling]

R6RS

\section{Pretest}

a ``pre-test'' for Scheme, so the reader can see if they need to read any of this

Even a reader who knows Scheme might want to read the pattern matching section

We also describe a few important differences between Scheme and Racket, to ensure the reader can use either one


%%%%%%%%%%%%%%% begin detailed Scheme description %%%%%%%%%%%%%%%%%
\section{Numbers}

In this book we restrict ourselves to non-negative integers, which may be of any size:

\verb|5|

\verb|42|

\verb|0|

\verb|37623489762387946782365476|

\section{Booleans}

The Boolean \verb|#f| represents ``false'', while
the Boolean \verb|#t| represents ``true''.

\section{\texttt{quote} and symbols}

In addition to numbers and Booleans, Scheme can represent abstract concepts and symbolic data using \emph{symbols}, sometimes called \emph{quoted symbols}.

If we want to create a symbol to represent the abstract concept of ``love'', we can write \verb|(quote love)| which produces the symbol \verb|love|.  Because symbols are used so often in Scheme, the equivalent shorthand notation \verb|'love| can also be used to produce the symbol \verb|love|.

\section{Expressions, values, and evaluation}

In Scheme terminology, \verb|(quote love)| is an \emph{expression}.
In Scheme, expressions are \emph{evaluated} to produce \emph{values}.
In this case, the expression \verb|(quote love)| evaluates to the value \verb|love|, which is a symbol. [todo consider pointing out that in Racket, by default, \verb|'love| will be displayed, and how to adjust that setting]

All Scheme symbols are values.  Numbers and the Booleans \verb|#f| and \verb|#t| are also values.

In Scheme we can also quote numbers and Booleans.
%
For example, the expression \verb|(quote 5)| evaluates to the value \verb|5|, which is a number.
%
Similarly, the expression \verb|(quote #f)| evaluates to the value \verb|#f|, which is a Boolean.

Actually, we don't need to quote numbers or Booleans in Scheme---numbers and Booleans are ``self-evaluating'' (or ``self-quoting'').  For example, the expression \verb|42| evaluates to the value \verb|42|, which is a number.  The expression \verb|#t| evaluates to the value \verb|#t|, which is a Boolean.  Scheme symbols, on the other hand, are not self-evaluating, and must be explicitly quoted.\footnote{Scheme symbols must be explicitly quoted so that they are distinct syntactically from variable references, which will encounter shortly.}

As shorthand, we write ``the expression \verb|(quote 5)| evaluates to the value \verb|5|'' as:

\verb|(quote 5) => 5|

\noindent
where the arrow \verb|=>| can be read as ``evaluates to''.

Similarly, we can write

\verb|(quote #f) => #f|

\verb|(quote love) => love|

\verb|'love => love|

\verb|(quote quote) => quote|

\verb|'quote => quote|

\verb|42 => 42|

\verb|6375764356 => 6375764356|

\noindent
and

\verb|#t => #t|.

\subsection{The general evaluation rule for \texttt{quote}}

We know that 

\verb|(quote 0) => 0|

\verb|(quote 1) => 1|

$\ldots$

\verb|(quote 42) => 42|

$\ldots$

\verb|(quote 3765783657849) => 3765783657849|

$\ldots$

\noindent
and so forth.

We can generalize our ``evaluates to'' \verb|=>| notation; the more general \emph{evaluation rule} for quoting numbers is:

\verb|(quote <num>) => <num>|

\noindent
for any number \verb|<num>|.  We use the name \verb|num| surrounded by the angle brackets \verb|<| and \verb|>| to represent any number.

Similarly, the evaluation rule for quoting Booleans is:

\verb|(quote <bool>) => <bool>|

\noindent
for any Boolean \verb|<bool>|.  (Of course there are only two Boolean values, \verb|#f| and \verb|#t|.)  And the evaluation rule for quoting symbols is 

\verb|(quote <sym>) => <sym>|

\noindent
for any symbol \verb|<sym>|.

More generally, the evaluation rule for \verb|quote| is:

\verb|(quote <datum>) => <datum>|

\noindent
The word \emph{datum} is the singular form of \emph{data}.  Numbers, Booleans, and symbols are three types of data we have encountered so far.

\section{\texttt{define}, definitions, and variables}

We can use \verb|define| to give a name to a value.

For example,

\verb|(define x 5)|

\noindent
gives the name \verb|x| to the number \verb|5|, while

\verb|(define cool-cat (quote Sugie))|

\noindent
gives the name \verb|cool-cat| to the symbol \verb|Sugie|.\footnote{Actually, in Scheme \texttt{(define x 5)} gives the name \texttt{x} to a \emph{location} that contains the value \texttt{5}.  It is possible to assign a different value to the location named by \texttt{x} using \texttt{set!}---for example, \texttt{(set! x 6)}.  We will avoid the use of \texttt{set!} for now, which means we pretend that \texttt{define} just gives a name to a value.}

More generally, we can write:

\verb|(define <id> <expr>)|

\noindent
where \verb|<id>| is any Scheme \emph{identifier} (such as \verb|x|, \verb|my-cat|, \verb|Hello_there=137^|, or \verb|関連-42|) and where \verb|<expr>| is any expression.

A use of \verb|define| is called a \emph{definition}. %
A definition is neither an expression nor a value---it is a \emph{statement}.  While evaluation of expressions produces values, statements are evaluated for their \emph{effects}.  The effect of evaluating \verb|(define x 5)| is to introduce a new \emph{variable} named \verb|x| that is \emph{bound} to the number \verb|5|.\footnote{Actually, the variable \texttt{x} is bound to a \emph{location} that contains the value \texttt{5}.}

Once we have defined a variable (such as \verb|x|), we can \emph{reference} (or \emph{refer to}) that variable to get the value to which it is bound (such as the number 5, in the case of the variable \verb|x|).

We can see the behavior of \verb|define| and variable reference at the Chez Scheme Read-Eval-Print Loop, or \emph{REPL}.  First we start Chez Scheme: 

\begin{alltt}
Chez Scheme Version 10.1.0
Copyright 1984-2024 Cisco Systems, Inc.

> 
\end{alltt}

\noindent
and then define \verb|x| to be \verb|5|:

\begin{alltt}
Chez Scheme Version 10.1.0
Copyright 1984-2024 Cisco Systems, Inc.

> (define x 5)
> 
\end{alltt}

The \verb|> | prompt on the line following \verb|> (define x 5)| indicates that Chez has evaluated the statement \verb|(define x 5)| and is ready to evaluate the next expression or statement.  To save space, we'll not show the \verb|> | prompt whenever an expression evaluates to a value that is printed at the REPL.

Now that we have defined the variable \verb|x|, we can refer to it:

\begin{alltt}
> x
5
\end{alltt}

\noindent
\verb|x| is an expression (a variable reference) that evaluates to the value \verb|5| (a number).

(In our arrow notation, we would write \verb|x => 5|.)

Let's define another variable, like we did above:

\begin{alltt}
> (define cool-cat (quote Sugie))
> 
\end{alltt}

\begin{alltt}
> cool-cat
Sugie
\end{alltt}

\noindent
\verb|cool-cat| is an expression (a variable reference) that evaluates to the value \verb|Sugie| (a symbol).

What happens if we refer to an \emph{unbound variable}---that is, a variable that has not been defined?

\begin{alltt}
> w

Exception: variable w is not bound
Type (debug) to enter the debugger. 
\end{alltt}

\noindent
Chez Scheme evaluates the expression \verb|w|, which is a variable reference.
Since \verb|w| is an unbound variable, Chez is not able to determine the value bound to \verb|w|.
Instead, Chez Scheme \emph{throws an exception} [todo check terminology: throw exception?] indicating that \verb|w| is a variable that is not bound.

Let's define \verb|w| to have the same value as does the variable \verb|x|:

\begin{alltt}
> (define w x)
> w
5
\end{alltt}

%% The \emph{expression} \verb|x| (which is a variable reference) \emph{evaluates} to the \emph{value} \verb|5| (which is a number).  [todo this sentence has a lot going on] [todo is there a way to introduce a compound expression by this point, so we don't just have variable reference?]

%% The expression \verb|cats-are-cool| (which is a variable reference) evaluates to the value \verb|#t| (which is a Boolean).

Recall the syntax for uses of \verb|define|:

\verb|(define <id> <expr>)|

\noindent
Also recall that \verb|(define x 5)| is a statement rather than an expression.
What happens if use a definition where an expression is required?  Chez Scheme complains by throwing a different type of exception:

\begin{alltt}
> (define z (define y 6))

Exception: invalid context for definition (define y 6)
Type (debug) to enter the debugger.
\end{alltt}

We have now encountered the crucial notions of Scheme expressions, values, and statements, which we will need in order to understand and write interpreters.\footnote{miniKanren also has the notions of expressions, values, and statements, and introduces the new notion of \emph{terms}, a generalization of the notion of values. [todo add crossref]}

\section{Procedures and procedure application}

What if we would like to add one to the number five?  One way to do this in Scheme is to write the expression \verb|(add1 5)|:

\begin{alltt}
> (add1 5)
6
\end{alltt}

%\verb|(add1 5) => 6|

The expression \verb|(add1 5)| is an example of a \emph{procedure application}, or just \emph{application} (sometimes called a \emph{procedure call}, or just \emph{call}).

If the expression \verb|(add1 5)| is a procedure application, then what is \verb|add1|?  Let's find out at the REPL:

\begin{alltt}
> add1
#<procedure add1>
\end{alltt}

Aha! 
\verb|add1| is a variable that is bound to a \emph{procedure}.
In Scheme procedures are values, just like numbers, Booleans, and symbols.
\verb|add1| is a \emph{built-in} procedure that is bound in Chez Scheme's \emph{initial environment}, which is the default set of variable bindings that exist when Chez is started.  We can extend or modify the initial environment, as we did before using \verb|define|.

We can nest procedure applications in Scheme:

\begin{alltt}
> (add1 (add1 5))
7 
\end{alltt}

There are many built-in procedures in the initial Scheme environment, and even more in the initial Chez Scheme environment, since Chez extends Scheme with many additional procedures.  If we want to add two numbers, we can use the built-in Scheme procedure \verb|+|:

\begin{alltt}
> (+ 3 4)
7
\end{alltt}

\noindent
and:

\begin{alltt}
> (+ 7835467856 98236472167)
106071940023
\end{alltt}

Of course, we can nest procedure applications:

\begin{alltt}
> (+ (add1 (+ 3 (add1 7) (+ 5 6))))
23
\end{alltt}

As is the case with \verb|add1|, \verb|+| is a variable that is bound to a procedure in Chez Scheme's initial environment:

\begin{alltt}
> +
#<procedure +>
\end{alltt}

The procedure bound to \verb|add1| takes exactly one argument.
In contrast, the procedure bound to \verb|+| is \emph{variadic}, meaning that it can take any number of arguments.  For example, the procedure bound to \verb|+| can take three arguments:

\begin{alltt}
> (+ 5 6 7)
18
\end{alltt}

\noindent
one argument:

\begin{alltt}
> (+ 5)
5
\end{alltt}

\noindent
or even zero arguments:

\begin{alltt}
> (+)
0
\end{alltt}

The expression \verb|(+)| evaluates to \verb|0| because zero is the additive identity (the number that when added to another number preserves the value of that second number).

\section{Predicates, including type predicates}

In Scheme, a \emph{predicate} is a procedure that, when called, always terminates (without signalling an error), and that always returns one of the two Boolean literals: \verb|#f| or \verb|#t|.

A \emph{type predicate} is a predicate that can be used to determine the type of a value.  For example, the predicate

\verb|number?|

\noindent
is a built-in procedure that determines whether its argument is a number:

\begin{alltt}
> number?
#<procedure number?>
> (number? 5)
#t
> (number? (+ 3 4))
#t
> (number? (quote cat))
#f
> (number? #t)
#f
> (number? number?)
#f
\end{alltt}

It is a Scheme convention to end the names of predicates with a question mark.
Also by convention, many people pronounce the \verb|?| at the end of the predicate's name as ``huh''; for example, \verb|number?| is pronounced ``number-huh''.

Scheme's built-in type predicates also include \verb|boolean?|, \verb|symbol?|, and \verb|procedure?|:

\begin{alltt}
> (boolean? #t)
#t
> (boolean? #f)
#t
> (boolean? 5)
#f
> (boolean? (+ 3 4))
#f
> (boolean? (quote cat))
#f
> (boolean? boolean?)
#f
> (boolean? (number? 5))
#t
> (boolean? (number? #f))
#t
\end{alltt}

\begin{alltt}
> (symbol? (quote cat))
#t
> (symbol? 5)
#f
> (symbol? (+ 3 4))
#f
> (symbol? #f)
#f
> (symbol? symbol?)
#f
> (symbol? (symbol? 5))
#f
\end{alltt}

\begin{alltt}
> (procedure? procedure?)
#t
> (procedure? +)
#t
> (procedure? add1)
#t
> (procedure? 5)
#f
> (procedure? (+ 3 4))
#f
> (procedure? (quote cat))
#f
> (procedure? (quote +))
#f
> (procedure? #t)
#f
> (procedure? (symbol? (quote cat)))
#f
\end{alltt}  


\section{\texttt{if}, test expressions, and truthiness}

We can make choices in Scheme using an \verb|if| expression, which is of the form

\verb|(if <test> <consequent> <alternative>)|

\noindent
where \verb|<test>|, \verb|<consequent>|, and \verb|<alternative>| are all expressions.
%
For example, in the expression \verb|(if #t 3 4)|, the \emph{subexpression} \verb|#t| is in the \emph{test position}, the subexpression \verb|3| is in the \emph{consequent position}, and the subexpression \verb|4| is in the \emph{alternative position}.

The rule for evaluation of an \verb|if| expression is that first the test subexpression is evaluated.  If the test subexpression evaluates to a true value, then the consequent subexpression is evaluated, and the resulting value is the value of the entire \verb|if| expression.  If the test subexpression evaluates to a false value, then the alternative subexpression is evaluated, and the resulting value is the value of the entire \verb|if| expression.

For example:

\begin{alltt}
> (if #t 3 4)
3
\end{alltt}
  
\begin{alltt}
> (if #f 3 4)
4
\end{alltt}

Of course we can use more complex expressions for the consequent and alternative subexpressions:

\begin{alltt}
> (if #t (+ 3 4) (+ 5 6))
7
> (if #f (+ 3 4) (+ 5 6))
11
\end{alltt}

\noindent
And we can use more complex expressions for the test subexpression:

\begin{alltt}
> (if (number? 72634786) (+ 3 4) (+ 5 6))
7
> (if (symbol? 72634786) (+ 3 4) (+ 5 6))
11
\end{alltt}

%% \verb|(if #t (+ 3 4) (+ 5 6)) => 7|

%% \verb|(if #f (+ 3 4) (+ 5 6)) => 11|

%% \verb|(if (number? 72634786) (+ 3 4) (+ 5 6)) => 7|

%% \verb|(if (symbol? 72634786) (+ 3 4) (+ 5 6)) => 11|

%\verb|(if <expr> <expr> <expr>)|

%\verb|(if <e1> <e2> <e3>)|


%\verb|(if <test> <conseq> <alt>)|

%\verb|(if <te> <ce> <ae>)|

%\verb|(if <t> <c> <a>)|

% one-armed vs. two-armed \verb|if|

\verb|#t| is not the only true value in Scheme.
In fact, \emph{any} value in Scheme other than \verb|#f| is considered true.
For example, both \verb|5| and \verb|0| are considered true values in Scheme.

\begin{alltt}
> (if 42 (+ 3 4) (+ 5 6))
7
> (if (* 6 7) (+ 3 4) (+ 5 6))
7
> (if 'cat (+ 3 4) (+ 5 6))
7
\end{alltt}

%% \verb|(if 42 (+ 3 4) (+ 5 6)) => 7|

%% \verb|(if 'cat (+ 3 4) (+ 5 6)) => 7|


\section{Evaluation order and special forms}

special forms vs. application

keywords

\verb|quote|, \verb|define|, and \verb|if| are keywords; \verb|(quote <datum>)|, \verb|(define <id> <expr>)|, and \verb|(if <expr> <expr> <expr>)| are special forms.

Recall that in the initial Scheme environment \verb|+| is a variable bound to a procedure that adds zero or more numbers:

\begin{alltt}
> +
#<procedure +>
\end{alltt}

In contrast, in the initial Scheme environment \verb|quote| is the \emph{keyword} of a special form.  Recall that \verb|(quote <datum>)| is the general syntax for a \verb|quote| expression.  The expression \verb|(quote cat)| evaluates to the symbol \verb|cat|.  However, evaluating the keyword \verb|quote| by itself leads to a \emph{syntax error}:

\begin{alltt}
> quote

Exception: invalid syntax quote
Type (debug) to enter the debugger.
\end{alltt}


\section{Comments}

% TODO These examples are a bit funny, since I'm only commenting out
% code, not text comments or whatever.  Maybe should change that up,
% or at least add extra examples.  And can include non-Latin Unicode,
% such as kanji for Kanren, in the comments.

Any characters on a line following the \verb|;| character will be ignored.  For example,

\verb|(* 3 4) ; (/ 5 0)|

\noindent
evaluates to \verb|12|.

The entire S-expression following \verb|#;| will be ignored.  For example,

\begin{alltt}
(+ 4 5) #;(* 6
             7)
\end{alltt}
  
Any characters between matching \texttt{\#|} and \texttt{|\#} will be igored.  For example:

% |#  ; <- added this so that Emacs doesn't think everything is commented!
                                        
% TODO nicer, more natural example
\begin{alltt}
(list
  (+ 3 4)
  #|
  erfjkhrj hfjk
  kjrhjkrheg rjghjer gj
  rghrejhj rjegh jrehk
  
  jehjkf klh fe
  |#
  (* 5 6)
  )
\end{alltt}

\noindent
is equivalent to

\begin{alltt}
(list
  (+ 3 4)
  (* 5 6)
  )
\end{alltt}

\noindent
which is equivalent to \verb|(list (+ 3 4) (* 5 6))|.

% TODO Explain that extra whitespace is ignored in Scheme (where to put this?)

% TODO Explain that, unlike in math notation and in many programming languages, in Scheme parens always ``mean something''---you can't add or remove them without changing the meaning of the program (as opposed to parens in math notation to just make implicit ordering of operations explicit, for example)

% TODO Should I mention Chez Scheme's non-standard #!eof  ?


\section{\texttt{cond}}

\section{A few other predicates}

\verb|zero?|

\verb|even?|

\verb|odd?|

\section{Lists}

\verb|list|

\verb|list?|

empty list (quoted)

\verb|null?|

quoted non-empty lists

nested lists

%% can nest quotes: even though the expressions 5 and (quote 5) both evaluate to the value 5, the expressions (quote (quote 5)) amd (quote 5) do not evaluate to the same value

%%   (quote cat)
%%   'cat

%%   (quote (quote cat))
%%   '(quote cat)

\section{Pairs and improper lists}

\verb|cons|

\verb|pair?|

\section{S-expressions}

[todo need to introduce the concept of the s-expression.  Now might be a good time, since we have symbols, numbers, booleans, pairs]

\section{\texttt{lambda}}


\section{Equality predicates}

\verb|=|

\verb|eq?|

\verb|equal?|

\section{Simple examples}

\subsection{\texttt{member?}}

\subsection{\texttt{length}}

\subsection{\texttt{append}}

\subsection{\texttt{assoc}}

\section{\texttt{let}}

\section{\texttt{letrec}}

\section{Lexical scope}

\section{More examples}

\subsection{\texttt{append} (\texttt{letrec} version)}

\subsection{\texttt{even?} and \texttt{odd?} (\texttt{define} version)}

\subsection{\texttt{even?} and \texttt{odd?} (\texttt{letrec} version)}

\subsection{Curried adder}

spelling of Curried?

\section{\texttt{eval}}

\section{Pattern matching}

\section{Grammar for our subset of Scheme}

\section{Differences between Scheme and Racket}

evaluation order

printed rep of quoted values

pattern mathing

require vs load

repl usage

\verb|eval| usage

\section{Exercises}



\chapter{A whirlwind introduction to relational programming in miniKanren}%%%%%%%%% intro to mk %%%%%%%%%%%%%%%%%%%%%

\section{What is relational programming?}

\section{Which version of miniKanren we are using, and why}

faster-miniKanren without defrel

\section{Useful miniKanren resources}

\section{Pretest}

someone who has read TRS1 or TRS2, or who has implemented microKanren,
still needs to know about \texttt{=/=}, \texttt{symbolo},
\texttt{numbero}, \texttt{absento}, and the differences between
miniKanren in those books and in this book

\section{miniKanren as an embedded DSL, and otherwise}

Scheme as host language

\section{Core miniKanren}

\subsection{\texttt{==}}

simlarity to equal? (but not to eq?)

first-order syntactic unification

\subsection{\texttt{run}$^{n}$}

\subsection{\texttt{conde}}

\subsection{\texttt{fresh}}

\subsection{\texttt{run*}}

\subsection{What miniKanren inherits from Scheme}

\section{Logic variables (or, what does ``variable'' even mean?)}

% \footnote{A word of caution regarding terminology.  As we will see shortly, miniKanren introduces the notion of \emph{logic variables}, which are different from Scheme variables.  We will use the terminology \emph{unassociated logic variable}}

\section{Expressions and terms}

\section{Groundness, and the parts of Scheme we can safely use}

\section{Relational vs. non-relational programming in miniKanren}

\section{Simple examples}

\subsection{\texttt{appendo}}

\subsection{\texttt{membero} (broken version)}

\section{Other useful constraints}

\subsection{\texttt{=/=}}

disequality

\subsection{\texttt{symbolo} and \texttt{numbero}}

not needed in OCanren, for example

\subsection{\texttt{absento}}

prevention of quoted closures (not needed in OCanren) and other uses,
such as \texttt{not-in-envo} in split env

\section{miniKanren Grammar}

beware nesting run or ==, calling Scheme eliminators, unifying with procedures, assuming a term is ground, assuming Scheme can handle even ground logic variables as values

revist in style and gotchas chapter

\section{More examples}

\subsection{\texttt{membero} (fixed version)}

\subsection{Differences between the miniKanren in this book and other miniKanrens}

TRS1

TRS2

microKanren

core.logic

OCanren

\subsection{Exercises}


\chapter{miniKanren style and common pitfalls}% mK style and pitfalls %%%%%%%%%%%%%%%%%%%%

``Will's Rule''

syntactic issue 1:
lambda (implicit begin) containing more than one goal expression (without a fresh wrapping those goals)---very hard to debug, since only one of the goals is actually run---defrel prevents this problem

syntactic issue 2:
nesting a goal expression inside of a call to ==---can actually succeed, although rarely does what you would intend

use of car, cdr, +, etc.

assuming a Scheme function can operate on the value of a ground logic variable

unifying with a Scheme procedure

mixing Scheme and mk code in a way that doesn't preserve relationality

incorrect tagging

\chapter{Debugging miniKanren code}%%%%%%%% debug mK %%%%%%%%%%%%%%%%%%%%%%%%%

\section{Debugging unexpected failure}

leave all args fresh

comment out clauses and goals

\section{Taming and debugging apparent divergence}

\verb|run 1| vs. \verb|run*|

run program with all arguments ground

reordering conjuncts

adding a depth counter

adding bounds (as in rel interp)

tabling

using occur check, presumably?

\section{Debugging interpreters (and interpreter-like programs)}

how to build up a \verb|conde|-based program, such as an interpreter, one expression at a time Dan Friedman-style and then run/test it

run program ``forward'' to test it

perhaps inclue alternative \verb|run| interafce/streaming/alternative set-based test macro


\chapter{A simple environment-passing Scheme interpreter in Scheme}% simple Scheme-in-Scheme %%%%%%%%%%

call-by-value (CBV) $\lambda$-calculus (variable reference, single-argument $lambda$, and procedure application), plus \verb|quote| and \verb|list|

association-list representation of the environment

empty initial environment

\verb|list| is implemented as if it were a special form rather than as a variable bound, in a non-empty initial environment, to a procedure.  As a result, although \verb|list| can be shadowed, \verb|(list list)| results in an error that there is an attempt to reference an unbound variable \verb|list|.

tagged list to represent closure

grammar for the language we are interpreting

\begin{alltt}
(load "pmatch.scm")
\end{alltt}

[TODO make sure I explain MIT vs Indiana syntax for \verb|define|]

\begin{alltt}
(define (eval expr)
  (eval-expr expr '()))
\end{alltt}

\begin{alltt}
(define (eval-expr expr env)
  (match expr
    ((quote ,v)
     (guard (not-in-env? 'quote env))
     v)
    ((list . ,e*)
     (guard (not-in-env? 'list env))
     (eval-list e* env))
    (,x
     (guard (symbolo? x))
     (lookup x env))
    ((,rator ,rand)
     (let ((a (eval-expr rand env)))
       (pmatch (eval-expr rator env)
         ((closure ,x ,body ,env^)
          (guard (symbol? x))
          (eval-expr body `((,x . ,a) . ,env^))))))
    ((lambda (,x) ,body)
     (guard (and (symbol? x)
                 (not-in-env? 'lambda env)))
     `(closure ,x ,body ,env))))
\end{alltt}

\begin{alltt}
(define (not-in-env? x env)
  (pmatch env
    (((,y . ,v) . ,env^)
     (if (equal? y x) ;; TODO eq? vs eqv? vs equal?, with equal? being semantically closest to ==
         #f
         (not-in-env? x env^)))
    (() #t))) ;; TODO empty env clause comes second; Dijkstra guard, and all that
\end{alltt}

\begin{alltt}
(define (eval-list expr env)
  (pmatch expr
    (() '())
    ((,a . ,d)
     (let ((t-a (eval-expr a env))
           (t-d (eval-list d env)))
       `(,t-a . ,t-d)))))
\end{alltt}

\begin{alltt}
(define (lookup x env)
  (pmatch env
    (() (error 'lookup "unbound variable")) ;; TODO make sure error is introduced, and make error message nicer
    (((,y . ,v) . ,env^)
     (if (equal? y x)
         v
         (lookup x env^)))))
\end{alltt}

\chapter{Rewriting the simple environment-passing Scheme interpreter in miniKanren}% simple Scheme-in-miniKanren %%%%%%%%%%

In this chapter we will translate the evaluator for the simple environment-passing interpreter from the previous chapter from a Scheme function to a miniKanren relation.

[TODO cite the code from the Quines interp in faster-miniKanren, and point to the 2012 SW paper on Quines]

[TODO this interp uses defrel---do I want to stick with defrel, or use define + lambda?  Or maybe the book shows both (probably needs to show both at some point)]

\begin{alltt}
(load "mk-vicare.scm")
(load "mk.scm")
\end{alltt}

\begin{alltt}
(defrel (evalo expr val)
  (eval-expro expr '() val))
\end{alltt}

\begin{alltt}
(defrel (eval-expro expr env val)
  (conde
    ((fresh (v)
       (== `(quote ,v) expr)
       (not-in-envo 'quote env)
       (absento 'closure v) ;; TODO discuss the tradeoffs of moving this absento to evalo
       (== v val)))
    ((fresh (e*)
       (== `(list . ,e*) expr)
       (not-in-envo 'list env)
       (absento 'closure e*) ;; TODO is this absento really needed, if we have absento for quote?
       (eval-listo e* env val)))
    ((symbolo expr) (lookupo expr env val))
    ((fresh (rator rand x body env^ a)
       (== `(,rator ,rand) expr)
       (eval-expro rator env `(closure ,x ,body ,env^))
       (eval-expro rand env a)
       (eval-expro body `((,x . ,a) . ,env^) val)))
    ((fresh (x body)
       (== `(lambda (,x) ,body) expr)
       (symbolo x)
       (not-in-envo 'lambda env)
       (== `(closure ,x ,body ,env) val)))))
\end{alltt}
  
\begin{alltt}
(defrel (not-in-envo x env)
  (conde
    ((fresh (y v env^)
       (== `((,y . ,v) . ,env^) env)
       (=/= y x)
       (not-in-envo x env^)))
    ((== '() env))))
\end{alltt}
  
\begin{alltt}
(defrel (eval-listo expr env val)
  (conde
    ((== '() expr)
     (== '() val))
    ((fresh (a d t-a t-d)
       (== `(,a . ,d) expr)
       (== `(,t-a . ,t-d) val)
       (eval-expro a env t-a)
       (eval-listo d env t-d)))))
\end{alltt}
  
\begin{alltt}
(defrel (lookupo x env t)
  (fresh (y v env^)
    (== `((,y . ,v) . ,env^) env)
    (conde
      ((== y x) (== v t))
      ((=/= y x) (lookupo x env^ t)))))
\end{alltt}



\chapter{Quine time}% Quine time %%%%%%%%%%%%%%%%%%%%%%

McCarthy challenge given in `A Micromanual for LISP'

\begin{alltt}
(run 1 (e) (evalo e e))
\end{alltt}
\verb|=>|
\begin{alltt}
((((lambda (_.0) (list _.0 (list 'quote _.0)))
   '(lambda (_.0) (list _.0 (list 'quote _.0))))
  (=/= ((_.0 closure)) ((_.0 list)) ((_.0 quote)))
  (sym _.0)))
\end{alltt}

\begin{alltt}
> ((lambda (_.0) (list _.0 (list 'quote _.0)))
    '(lambda (_.0) (list _.0 (list 'quote _.0))))
((lambda (_.0) (list _.0 (list 'quote _.0)))
  '(lambda (_.0) (list _.0 (list 'quote _.0))))
\end{alltt}

We replace \verb|_.0| with the arbitrary free variable name \verb|x| to produce the canonical LISP/Scheme Quine:

\begin{alltt}
((lambda (x) (list x (list 'quote x)))
 '(lambda (x) (list x (list 'quote x))))
\end{alltt}

\begin{alltt}
> ((lambda (x) (list x (list 'quote x)))
   '(lambda (x) (list x (list 'quote x))))
((lambda (x) (list x (list 'quote x)))
  '(lambda (x) (list x (list 'quote x))))
\end{alltt}

Twines

every Quine is trivially a Twine; we can add a disequality constraint to ensure \verb|p| and \verb|q| are distinct terms

\begin{alltt}
> (run 1 (p q)
    (=/= p q)
    (evalo p q)
    (evalo q p))
[TODO add the answer]
\end{alltt}

\footnote{I thank Larry Moss and the Indiana University Logic Symposium [TODO check the name of the symposium] for inviting me to give a talk where I demonstrated Quine generation, and where Larry suggested I tried generating Twines.}

Thrines

\begin{alltt}
> (run 1 (p q r)
    (=/= p q)
    (=/= p r)
    (=/= q r)    
    (evalo p q)
    (evalo q r)
    (evalo r p))
[TODO add the answer]
\end{alltt}


Structurally boring Quines, Twines, and Thrines

just moving quotes around

\verb|absento| trick to generate more interesting Quines, Twines, and Thrines

\begin{alltt}
> (run 1 (p q)
    (absento p q)
    (absento q p)    
    (evalo p q)
    (evalo q p))
[TODO add the answer]
\end{alltt}

[similarly for Thrines]

Revisiting our original Quine query with the \verb|absento| trick

\begin{alltt}
(run 1 (p)
  (fresh (expr1 expr2)
    (absento expr1 expr2)
    (== `(,expr1 . ,expr2) p)
    (evalo p p)))
\end{alltt}
\verb|=>|
\begin{alltt}
((((lambda (_.0)
     (list (list 'lambda '(_.0) _.0) (list 'quote _.0)))
   '(list (list 'lambda '(_.0) _.0) (list 'quote _.0)))
  (=/= ((_.0 closure)) ((_.0 list)) ((_.0 quote)))
  (sym _.0)))
\end{alltt}


\chapter{Using a two-list representation of the environment}% split environment representation

association-list representation of an environment where \verb|x| is mapped to the list \verb|(cat dog)| and \verb|y| is mapped to \verb|5|:
\begin{alltt}
((x . (cat dog))
 (y . 5))
\end{alltt}
  
``split'' two-list representation of the same environment:
\begin{alltt}
(x y) ; variables

((cat dog) 6) ; values
\end{alltt}

;; a-list env
;; ((x . (cat dog))
;;  (y . 5))

;; split env
;; (x y)
;; ((cat dog) 6)

\verb|absento| trick for lazy \verb|not-in-envo|

;; (absento 'closure expr)

;; (absento t1 t2)

;; (not-in-envo 'lambda env)
;; (absento 'lambda '(x y))

[TODO do I want to make eval-expro representation-independent wrt environments?]

\begin{alltt}
(defrel (evalo expr val)
  (eval-expro expr '(() . ()) val))
\end{alltt}

\begin{alltt}  
(defrel (eval-expro expr env val)
  (conde
    ;; quote, list, and variable reference/lookup clauses elided
    ((fresh (rator rand body env^ a x x* v*)
       (== `(,rator ,rand) expr)
       (== `(,x* . ,v*) env^)
       (eval-expro rator env `(closure ,x ,body ,env^))
       (eval-expro rand env a)
       (eval-expro body
                  `((,x . x*) . (,a . v*))
                  val)))
    ;; lambda clause elided
    ))
\end{alltt}

\begin{alltt}
(defrel (not-in-envo x env)
  (fresh (x* v*)
    (== `(,x* . ,v*) env)
    (absento x x*)))
\end{alltt}

[TODO discuss tradeoffs between asserting (symbolo x) in these helper relations---how stand-alone do we want them?]

\begin{alltt}
(defrel (lookupo x env t)
  (fresh (y x* v v*)
    (== `((,y . ,x*) . (,v . ,v*)) env)
    (conde
      ((== y x) (== v t))
      ((=/= y x) (lookupo x `(,x* . ,v*) t)))))
\end{alltt}


\chapter{Extending the interpreter to handle \texttt{append}}% extended append interpreter

multi-argument and variadic \verb|lambda| and application 


\chapter{Using a non-empty initial environment}%

\verb|cons|, \verb|car|, \verb|cdr|, and \verb|null?| bound in the initial env to prims

\verb|list| bound in the initial env to the closure that results from evaluating the variadic \verb|(lambda x x)|


\chapter{Adding explicit errors}% explicit errors


\chapter{Adding delimited control operators}%

delimited continuations and/or effect handlers---can we do so in such a way that avoids ``breaking the wires''?

talk about the problem with \verb|call/cc| and breaking the wires


\chapter{Adding mutation}% mutation

support \verb|set!| (can we get away with supporting \verb|set!| without adding a store?)

support mutiple pairs and have an explicit store


\chapter{Writing a parser as a relation}% relational parser


\chapter{Writing a type inferencer as a relation}% relational type inferencer


\chapter{Build your own Barliman}% build your own Barliman


\chapter{Speeding up the interpreter}% speeding up the interpreter

[restrict to interpreter changes that don't require hacking faster-miniKanren or in-depth knowledge of the implementation]

dynamic reordering of conjuncts, especially for application

fast environment lookup for environments that are sufficiently ground


\chapter{Open problems}% open problems 


\appendix


\end{document}
