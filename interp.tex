\documentclass{book}

\usepackage{alltt}
\usepackage{calc}
\usepackage{ccicons} % Creative Commons icons
\usepackage[colorlinks,linktocpage]{hyperref}
\usepackage{url}
\usepackage{xeCJK}

\setCJKmainfont{BIZ UDMincho}
\setCJKsansfont{BIZ UDGothic}
\setCJKmonofont{BIZ UDGothic}

\newlength{\negspacelength}
\setlength{\negspacelength}{\bigskipamount+\medskipamount}

\newcommand{\negspace}{\vspace{-\negspacelength}}

\def\whalfspace{\vspace*{0.335\baselineskip}}
\def\wspace{\vspace*{0.67\baselineskip}}
\def\nspace{\vspace*{-.7\baselineskip}}
\def\nspacemargin{\vspace*{-1\baselineskip}}

\begin{document}

\frontmatter

\date{\today}

\title{Relational Interpreters in miniKanren \\ \ \\ (WORKING ROUGH DRAFT --- DRAFT 0)}

\author{William E. Byrd}

\maketitle

{
  \fontsize{12}{12}
  \mbox{}
  \vfill

%  \hrule

\noindent
\textcopyright~2024 William E. Byrd


\noindent
\huge
\ccLogo
\ccAttribution
\large


\noindent
This work is licensed under a Creative Commons Attribution 4.0 International License.
(CC BY 4.0) \\
\url{http://creativecommons.org/licenses/by/4.0/}
}

\newpage

{
\ \\

\ \\

\noindent
\huge
To Dan Friedman
}

\newpage

\tableofcontents

\chapter{Preface}

The intent of this book is to share the techniques, knowledge, pitfalls, open problems, promising-looking future work/techniques, and literature of writing interpreters as relations in miniKanren.
%
Someone who reads this book actively should be ready to understand, implement, modify, and improve interpreters wtitten as miniKanren relations, read the related literature, and perform original research on the topic.

\section{What this book is about}

This book is about writing interpreters for programming languages, especially for subsets of Scheme.
While there are many books on writing interpreters, this book is unusual in that it explores
how to write interpreters as \emph{relations} in the miniKanren relational programming language.
By writing interpreters as relations, and by using the implicit constraint solving and search in the \texttt{faster-miniKanren} implementation, we can use the flexibility of relational programming to allow us to experiment with programs in the language being interpreted.  For example, a relational interpreter can interpret a program with missing subexpressions\footnote{Such programs are often called \emph{program sketches} [TODO cite].}, or \emph{holes}, attempting to fill in the missing subexpressions with values that result in valid programs in the language being interpreted.  Or we can give both a program containing holes and the value we expect the program to produce when interpreted, and let \texttt{faster-miniKanren} try to fill in the holes in a way to produce the expected output.  We can even write an interpreter that explicitly handles errors, and ask \texttt{faster-miniKanren} to find inputs to the program that trigger these errors.\footnote{This is known in the literature as ``angelic execution''.}

\section{What you need to know to read this book}

[TODO add topics that the reader should know, including Scheme, miniKanren, lexical scope, environment-passing interpreters, etc]

[TODO add pointers to resources]

\section{Running the code in this book}

The code in this book was tested with Chez Scheme and Racket.  It should be possible to run most code in other Scheme implementations, with few or no changes, with the exception of code that makes extensive use of Chez-specific or Racket-specific features, which I will point out in those chapters, as appropriate.

\subsection{Getting \texttt{pmatch} from GitHub}

\subsection{Getting \texttt{faster-miniKanren} from GitHub}

\url{https://github.com/michaelballantyne/faster-miniKanren}

\begin{alltt}
git clone git@github.com:michaelballantyne/faster-minikanren.git
\end{alltt}

Alternatively, you can click on \verb|<> Code| button and select \verb|Download ZIP| to download and uncompress the \texttt{.zip} file containing the entire \texttt{faster-miniKanren} directory.


\subsection{Using this book with Chez Scheme}

\subsubsection{Installing Chez Scheme}

\subsubsection{Starting a Chez Scheme REPL}

\subsubsection{Loading a file in Chez Scheme}

\subsubsection{Loading \texttt{faster-miniKanren} in Chez Scheme}


\subsection{Using this book with Racket}

\subsubsection{Installing Racket}

\url{https://racket-lang.org/}

\url{https://download.racket-lang.org/}

\subsubsection{Important differences between Chez Scheme and Racket}

representation of quoted values

evaluation order

language levels

macros

\subsubsection{The DrRacket IDE and the Racket REPL}

\subsubsection{Starting and configuring DrRacket}

changing default language

changing default memory limit

\subsubsection{Starting a Racket REPL}

\subsubsection{Requiring a module in Racket}

\subsubsection{Requiring the \texttt{faster-miniKanren} module in Racket}

\section{Acknowledgements}

Dan Friedman and Michael Ballantyne both encouraged me to continue working on this book, and independently encouraged me to break down one giant book into more than one book, each book being more manageable.
%
Both Dan and Michael encouraged me to avoid getting bogged down with a lot of introductory material, which had caused me to abandon previous writing efforts.
%
Michael also encouraged me to continue working on the book in the open.

Darius Bacon wrote me a very helpful email about how using two separate lists to represent a lexical environment, rather than a single association list, can result in better performance and divergence behavior.  I had played around with this representation in the past, but had abandoned it before I understood its advantages.  Thank you, Darius.

My mother has continually encouraged me to work on this book, and most importantly, to finish it!

[TODO add other acknowledgements]

[TODO add acks for typesetting tech, such as the fonts; also can add colophon if I'm so inclined]

\mainmatter

\chapter{A simple environment-passing Scheme interpreter in Scheme}% simple Scheme-in-Scheme %%%%%%%%%%

call-by-value (CBV) $\lambda$-calculus (variable reference, single-argument $lambda$, and procedure application), plus \verb|quote| and \verb|list|

association-list representation of the environment

empty initial environment

\verb|list| is implemented as if it were a special form rather than as a variable bound, in a non-empty initial environment, to a procedure.  As a result, although \verb|list| can be shadowed, \verb|(list list)| results in an error that there is an attempt to reference an unbound variable \verb|list|.

tagged list to represent closure

grammar for the language we are interpreting

\begin{alltt}
(load "pmatch.scm")
\end{alltt}

[TODO make sure I explain MIT vs Indiana syntax for \verb|define|]

\begin{alltt}
(define (eval expr)
  (eval-expr expr '()))
\end{alltt}

\begin{alltt}
(define (eval-expr expr env)
  (pmatch expr
    ((quote ,v)
     (guard (not-in-env? 'quote env))
     v)
    ((list . ,e*)
     (guard (not-in-env? 'list env))
     (eval-list e* env))
    (,x
     (guard (symbolo? x))
     (lookup x env))
    ((,rator ,rand)
     (let ((a (eval-expr rand env)))
       (pmatch (eval-expr rator env)
         ((closure ,x ,body ,env^)
          (guard (symbol? x))
          (eval-expr body `((,x . ,a) . ,env^))))))
    ((lambda (,x) ,body)
     (guard (and (symbol? x)
                 (not-in-env? 'lambda env)))
     `(closure ,x ,body ,env))))
\end{alltt}

\begin{alltt}
(define (not-in-env? x env)
  (pmatch env
    (((,y . ,v) . ,env^)
     (if (equal? y x) ;; TODO eq? vs eqv? vs equal?, with equal? being semantically closest to ==
         #f
         (not-in-env? x env^)))
    (() #t))) ;; TODO empty env clause comes second; Dijkstra guard, and all that
\end{alltt}

\begin{alltt}
(define (eval-list expr env)
  (pmatch expr
    (() '())
    ((,a . ,d)
     (let ((t-a (eval-expr a env))
           (t-d (eval-list d env)))
       `(,t-a . ,t-d)))))
\end{alltt}

\begin{alltt}
(define (lookup x env)
  (pmatch env
    (() (error 'lookup "unbound variable")) ;; TODO make sure error is introduced, and make error message nicer
    (((,y . ,v) . ,env^)
     (if (equal? y x)
         v
         (lookup x env^)))))
\end{alltt}

\chapter{Rewriting the simple environment-passing Scheme interpreter in miniKanren}% simple Scheme-in-miniKanren %%%%%%%%%%

In this chapter we will translate the evaluator for the simple environment-passing interpreter from the previous chapter from a Scheme function to a miniKanren relation.

[TODO cite the code from the Quines interp in faster-miniKanren, and point to the 2012 SW paper on Quines]

[TODO this interp uses defrel---do I want to stick with defrel, or use define + lambda?  Or maybe the book shows both (probably needs to show both at some point)]

\begin{alltt}
(load "mk-vicare.scm")
(load "mk.scm")
\end{alltt}

\begin{alltt}
(defrel (evalo expr val)
  (eval-expro expr '() val))
\end{alltt}

\begin{alltt}
(defrel (eval-expro expr env val)
  (conde
    ((fresh (v)
       (== `(quote ,v) expr)
       (not-in-envo 'quote env)
       (absento 'closure v) ;; TODO discuss the tradeoffs of moving this absento to evalo
       (== v val)))
    ((fresh (e*)
       (== `(list . ,e*) expr)
       (not-in-envo 'list env)
       (absento 'closure e*) ;; TODO is this absento really needed, if we have absento for quote?
       (eval-listo e* env val)))
    ((symbolo expr) (lookupo expr env val))
    ((fresh (rator rand x body env^ a)
       (== `(,rator ,rand) expr)
       (eval-expro rator env `(closure ,x ,body ,env^))
       (eval-expro rand env a)
       (eval-expro body `((,x . ,a) . ,env^) val)))
    ((fresh (x body)
       (== `(lambda (,x) ,body) expr)
       (symbolo x)
       (not-in-envo 'lambda env)
       (== `(closure ,x ,body ,env) val)))))
\end{alltt}
  
\begin{alltt}
(defrel (not-in-envo x env)
  (conde
    ((fresh (y v env^)
       (== `((,y . ,v) . ,env^) env)
       (=/= y x)
       (not-in-envo x env^)))
    ((== '() env))))
\end{alltt}
  
\begin{alltt}
(defrel (eval-listo expr env val)
  (conde
    ((== '() expr)
     (== '() val))
    ((fresh (a d t-a t-d)
       (== `(,a . ,d) expr)
       (== `(,t-a . ,t-d) val)
       (eval-expro a env t-a)
       (eval-listo d env t-d)))))
\end{alltt}
  
\begin{alltt}
(defrel (lookupo x env t)
  (fresh (y v env^)
    (== `((,y . ,v) . ,env^) env)
    (conde
      ((== y x) (== v t))
      ((=/= y x) (lookupo x env^ t)))))
\end{alltt}



\chapter{Quine time}% Quine time %%%%%%%%%%%%%%%%%%%%%%

McCarthy challenge given in `A Micromanual for LISP'

\begin{alltt}
(run 1 (e) (evalo e e))
\end{alltt}
\verb|=>|
\begin{alltt}
((((lambda (_.0) (list _.0 (list 'quote _.0)))
   '(lambda (_.0) (list _.0 (list 'quote _.0))))
  (=/= ((_.0 closure)) ((_.0 list)) ((_.0 quote)))
  (sym _.0)))
\end{alltt}

\begin{alltt}
> ((lambda (_.0) (list _.0 (list 'quote _.0)))
    '(lambda (_.0) (list _.0 (list 'quote _.0))))
((lambda (_.0) (list _.0 (list 'quote _.0)))
  '(lambda (_.0) (list _.0 (list 'quote _.0))))
\end{alltt}

We replace \verb|_.0| with the arbitrary free variable name \verb|x| to produce the canonical LISP/Scheme Quine:

\begin{alltt}
((lambda (x) (list x (list 'quote x)))
 '(lambda (x) (list x (list 'quote x))))
\end{alltt}

\begin{alltt}
> ((lambda (x) (list x (list 'quote x)))
   '(lambda (x) (list x (list 'quote x))))
((lambda (x) (list x (list 'quote x)))
  '(lambda (x) (list x (list 'quote x))))
\end{alltt}

Twines

every Quine is trivially a Twine; we can add a disequality constraint to ensure \verb|p| and \verb|q| are distinct terms

\begin{alltt}
> (run 1 (p q)
    (=/= p q)
    (evalo p q)
    (evalo q p))
[TODO add the answer]
\end{alltt}

\footnote{I thank Larry Moss and the Indiana University Logic Symposium [TODO check the name of the symposium] for inviting me to give a talk where I demonstrated Quine generation, and where Larry suggested I tried generating Twines.}

Thrines

\begin{alltt}
> (run 1 (p q r)
    (=/= p q)
    (=/= p r)
    (=/= q r)    
    (evalo p q)
    (evalo q r)
    (evalo r p))
[TODO add the answer]
\end{alltt}


Structurally boring Quines, Twines, and Thrines

just moving quotes around

\verb|absento| trick to generate more interesting Quines, Twines, and Thrines

\begin{alltt}
> (run 1 (p q)
    (absento p q)
    (absento q p)    
    (evalo p q)
    (evalo q p))
[TODO add the answer]
\end{alltt}

[similarly for Thrines]

Revisiting our original Quine query with the \verb|absento| trick

\begin{alltt}
(run 1 (p)
  (fresh (expr1 expr2)
    (absento expr1 expr2)
    (== `(,expr1 . ,expr2) p)
    (evalo p p)))
\end{alltt}
\verb|=>|
\begin{alltt}
((((lambda (_.0)
     (list (list 'lambda '(_.0) _.0) (list 'quote _.0)))
   '(list (list 'lambda '(_.0) _.0) (list 'quote _.0)))
  (=/= ((_.0 closure)) ((_.0 list)) ((_.0 quote)))
  (sym _.0)))
\end{alltt}


\chapter{Using a two-list representation of the environment}% split environment representation

association-list representation of an environment where \verb|x| is mapped to the list \verb|(cat dog)| and \verb|y| is mapped to \verb|5|:
\begin{alltt}
((x . (cat dog))
 (y . 5))
\end{alltt}
  
``split'' two-list representation of the same environment:
\begin{alltt}
(x y) ; variables

((cat dog) 6) ; values
\end{alltt}

;; a-list env
;; ((x . (cat dog))
;;  (y . 5))

;; split env
;; (x y)
;; ((cat dog) 6)

\verb|absento| trick for lazy \verb|not-in-envo|

;; (absento 'closure expr)

;; (absento t1 t2)

;; (not-in-envo 'lambda env)
;; (absento 'lambda '(x y))

\begin{alltt}
(defrel (evalo expr val)
  (eval-expro expr '(() . ()) val))
\end{alltt}

\begin{alltt}  
(defrel (eval-expro expr env val)
  (conde
    ;; quote, list, and variable reference/lookup clauses elided
    ((fresh (rator rand body env^ a x x* v*)
       (== `(,rator ,rand) expr)
       (== `(,x* . ,v*) env^)
       (eval-expro rator env `(closure ,x ,body ,env^))
       (eval-expro rand env a)
       (eval-expro body
                  `((,x . x*) . (,a . v*))
                  val)))
    ;; lambda clause elided
    ))
\end{alltt}

\begin{alltt}
(defrel (not-in-envo x env)
  (fresh (x* v*)
    (== `(,x* . ,v*) env)
    (absento x x*)))
\end{alltt}

[TODO discuss tradeoffs between asserting (symbolo x) in these helper relations---how stand-alone do we want them?]

\begin{alltt}
(defrel (lookupo x env t)
  (fresh (y x* v v*)
    (== `((,y . ,x*) . (,v . ,v*)) env)
    (conde
      ((== y x) (== v t))
      ((=/= y x) (lookupo x `(,x* . ,v*) t)))))
\end{alltt}



\chapter{Extending the interpreter to handle \texttt{append}}% extended append interpreter

add \verb|cons|, \verb|car|, \verb|cdr|, \verb|null?|, and \verb|if|

extend \verb|lambda| and application to handle multiple arguments and variadic 


\chapter{Using a non-empty initial environment}%

new case to handle prim app rather than user-defined closure app

\verb|cons|, \verb|car|, \verb|cdr|, and \verb|null?| bound in the initial env to prims

\verb|list| bound in the initial env to the closure that results from evaluating the variadic \verb|(lambda x x)|


\chapter{Adding explicit errors}% explicit errors


\chapter{Adding delimited control operators}%

delimited continuations and/or effect handlers---can we do so in such a way that avoids ``breaking the wires''?

talk about the problem with \verb|call/cc| and breaking the wires


\chapter{Adding mutation}% mutation

support \verb|set!| (can we get away with supporting \verb|set!| without adding a store?)

support mutiple pairs and have an explicit store


\chapter{Writing a parser as a relation}% relational parser


\chapter{Writing a type inferencer as a relation}% relational type inferencer


\chapter{Build your own Barliman}% build your own Barliman


\chapter{Speeding up the interpreter}% speeding up the interpreter

[restrict to interpreter changes that don't require hacking faster-miniKanren or in-depth knowledge of the implementation]

dynamic reordering of conjuncts, especially for application

fast environment lookup for environments that are sufficiently ground


\chapter{Open problems}% open problems 


\appendix


\end{document}
